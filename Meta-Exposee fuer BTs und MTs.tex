\documentclass[11pt]{scrartcl}

%---------------------------------
% Sprache, Schriften, Zeichensatz
%---------------------------------
\usepackage[ngerman]{babel}

\usepackage[T1]{fontenc}
\usepackage[utf8]{inputenc}

\usepackage{csquotes}	% für babel

%---------------------------------
% Datum/Zeit
%---------------------------------
\usepackage[ngerman]{datetime}

\newdateformat{germandate}{\THEDAY. \monthname[\THEMONTH] \THEYEAR}


%---------------------------------
% BibLaTeX: Online-Quellen
%---------------------------------
\usepackage[backend=biber, style=numeric, sorting=none]{biblatex}
% sorting=none -> keine Sortierung, Standard ist alphabetisch

% TexMaker-Kommando für bib(la)tex: "biber" %
% (Original-Einstellung für bibtex: "bibtex" %.aux

\bibliography{literatur}

%---------------------------------
% Meta Variables
%---------------------------------
\newcommand{\MetaInstitute}{Hochschule Bremen}
\newcommand{\MetaUnit}{Fakultät 4 -- Elektrotechnik und Informatik}
\newcommand{\MetaTitle}{ Human Centred Design und agile Entwicklungsmethoden}
\newcommand{\MetaSubtitle}{Durchführung und Evaluation anhand der MyCompetence Applikation der Lufthansa Industry Solution TS GmbH}
\newcommand{\MetaTask}{Exposé}
\newcommand{\MetaAuthorName}{Dr. Andreas Teufel}
\newcommand{\MetaAuthorSurname}{Teschke}
\newcommand{\MetaAuthor}{\MetaAuthorName~\MetaAuthorSurname}
\newcommand{\MetaStudentNumber}{\textit{5034226}}
\newcommand{\MetaStudyProgram}{Internationaler Studiengang Medieninformatik (B.Sc.)}
\newcommand{\MetaCoAuthor}{mit Prof. Dr. Martin Hering}
%\newcommand{\MetaDate}{\germandate\today}
\newcommand{\MetaDate}{5.\ September 2018}
\newcommand{\MetaVersion}{1.01}


%---------------------------------
% Querverweise
%---------------------------------
% Hyperref (sollte unbedingt vor geometry-Paket geladen werden)
% für Anzeige in Acrobat (pdfstartview)
\usepackage[bookmarks=false,
    pdfstartview=FitV,
    pdfhighlight={/I},
	colorlinks = true,
	linkcolor = blue,
	urlcolor  = blue,
	citecolor = blue,
    pdfborder={0 0 0},
    german]{hyperref}

\hypersetup{
  pdfauthor   = {\MetaAuthor},
  pdftitle    = {\MetaTitle},
  % pdfsubject  = {\MetaUnit, \MetaTask},
  pdfsubject  = {\MetaTask},
  pdfkeywords = {\MetaTitle, \MetaUnit, \MetaInstitute},
  % pdfkeywords = {\MetaTitle, \MetaUnit, \MetaTask, \MetaInstitute},
}

% Für Verweise mit Angabe des Typs des referenzierten Objekts (z.B. "Kapitel")
\usepackage[ngerman]{cleveref}

\crefname{section}{Kapitel}{Kapitel}	% Anpassung der Typbezeichnungen
\crefname{subsection}{Abschnitt}{Abschnitte}
\crefname{lstlisting}{Listing}{Listing} % neue Typbezeichnung


%---------------------------------
% Grafiken, Farben
%---------------------------------
\usepackage{graphicx}
\graphicspath{ {figures/} }

\usepackage{xcolor}

\definecolor{keyword}{HTML}{0000FF}	% Farben für Listings
\definecolor{string}{HTML}{D12F1B}
\definecolor{comment}{HTML}{008400}
\definecolor{lightgrey}{rgb}{0.99,0.99,0.99}

\definecolor{note}{rgb}{1,1,0.8}		% Farbe für Notizen


%---------------------------------
% Seitenlayout: Abmessungen
%---------------------------------
% Option für zusätzlichen Rand zum Binden: bindingoffset
% Default-Verhältnis von oberem (innerem) zu unterem (äußerem) Rand: 2:3
%\usepackage[top=3.6cm,bottom=4.50cm,left=2.3cm,bindingoffset=0.5cm, pdftex,twoside,a4paper]{geometry}
\usepackage[top=3.6cm,bottom=4.50cm,left=2.3cm,bindingoffset=0.5cm, pdftex,a4paper]{geometry}

\setlength{\parindent}{6mm} 
\setlength{\parskip}{0.2cm} 


% Raum für Notizen mit note-Paket
%\usepackage[top=3.6cm,bottom=4.5cm,right=4cm,bindingoffset=0.5cm,twoside]{geometry}

% "includemp" zieht Notizbereich (marginpar) von Druckbereich ab:
%\usepackage[top=3.6cm,bottom=4.50cm,left=2.3cm,bindingoffset=0.5cm,marginparwidth=3cm, includemp, pdftex,twoside,a4paper]{geometry}

%---------------------------------
% Seitenlayout: Kopf- und Fußzeilen
%---------------------------------
\usepackage[headsepline]{scrlayer-scrpage}

\clearpairofpagestyles
\automark[section]{section}	% Anzeige der "section" in der Kopfzeile
\lohead{\headmark}
\cofoot[\pagemark]{\pagemark} % Anzeige der Seitenzahl in der Fußzeile
\pagestyle{scrheadings}

%\pagestyle{empty}


%---------------------------------
% Seitenlayout: Überschriften
%---------------------------------
\addtokomafont{subsubsection}{\normalfont\sffamily}	% subsubsection nicht bold
\addtokomafont{paragraph}{\normalfont\sffamily}	% paragraph nicht bold


%---------------------------------
% Enumeration mit Zahlen auf allen Ebenen (für Gliederung)
%---------------------------------
\usepackage{enumitem}

\newlist{gliederung}{enumerate}{4}
\setlist[gliederung]{label*=\arabic*.}


%---------------------------------
% Listings
%---------------------------------
\usepackage{listings}

\lstset{
  language=Java,
  basicstyle=\ttfamily,
  showstringspaces=false, % lets spaces in strings appear as real spaces
  columns=fixed,
  keepspaces=true,
  keywordstyle=\color{keyword},
  stringstyle=\color{string},
  commentstyle=\color{comment},
  frame=tb,	% Rahmen = single
%  framerule=1pt,
  showstringspaces=false,
  basicstyle=\footnotesize\ttfamily,
  backgroundcolor=\color{lightgrey},
  numbers=left
}


%---------------------------------
% To Do Notes
%---------------------------------
\usepackage[textwidth=3.5cm, backgroundcolor=note]{todonotes}


%---------------------------------
% Tabellen
%---------------------------------
\usepackage{multirow}
%usepackage{array}
\usepackage{makecell}
\usepackage{booktabs}	% http://ctan.org/pkg/booktabs

\newcommand{\tabitem}{~~\llap{\textbullet}~~}


%---------------------------------
% Floating environments genauer positionieren
%---------------------------------
\usepackage{float}

%---------------------------------
% Wasserzeichen
%---------------------------------
%\usepackage{draftwatermark}
%
%\SetWatermarkText{ENTWURF}
%\SetWatermarkScale{2.5}


%---------------------------------
% Commands
%---------------------------------
\newcommand{\HRule}{\rule{\linewidth}{0.2mm}}	% Horizontal line for title page
\newcommand{\qto}[1]{\glqq #1\grqq}				% Anführungszeichen

\newcommand{\urlMitUmlauten}[2]{\texttt{\href{#1}{#2}}}				% URL mit Umlauten

% Abkürzungen (mit korrekten Abständen)
\usepackage{xspace}
\newcommand{\zB}{\mbox{z.\,B.}\xspace}
\newcommand{\dH}{\mbox{d.\,h.}\xspace} % Kommando \dh ist schon definiert
\newcommand{\ggf}{ggf.\xspace}
\newcommand{\evtl}{evtl.\xspace}
\newcommand{\bzw}{bzw.\xspace}


%---------------------------------
% Document start
%---------------------------------
\begin{document}

%---------------------------------
% Titlepage
%---------------------------------
\begin{titlepage}
  	\shortdate % Use Short Date
  	\center % Center everything on the page

  	~\\[1cm]

	%---------------------------------
	% HEADER SECTIONS
	%---------------------------------

	\begin{figure}[h!]
    		\centering
    		\resizebox{10cm}{!}{
      		\includegraphics[width=90mm,keepaspectratio]{HSB_Horizontal_RGB}
    		}
	\end{figure}

	\vspace{-0.5cm}
	\textsc{\Large \MetaInstitute}\\[0.2cm] % Major heading such as course name
	\textsc{\Large \MetaUnit}%[2.5cm] % Major heading such as course name
	
	\textsc{\large \MetaStudyProgram}\\[1.5cm]
	
	%---------------------------------
	% DOCUMENT TYPE SECTION
	%---------------------------------
	\textsc{\LARGE \MetaTask}\\[1.5cm] % Minor heading such as course title

	%---------------------------------
	% TITLE SECTION
	%---------------------------------
%	\HRule \\[0.5cm]
%	{
%		\LARGE \bfseries \MetaTitle \\[0.50cm] % Title of your document
%		\par
%	}
%	\HRule \\[1.5cm]
	\HRule \\[0.5cm]
	{
		\LARGE \bfseries \MetaTitle \\[0.50cm] % Title of your document
		\Large \bfseries -- \MetaSubtitle\ -- \\[0.50cm] % Title of your document
		\par
	}
	\HRule \\[1.5cm]

	%---------------------------------
	% AUTHOR SECTION
	%---------------------------------
	\large 
	\MetaAuthor\ (\MetaStudentNumber)\\
 	\MetaCoAuthor\\[0.25cm]

	%---------------------------------
	% DATE SECTION
	%---------------------------------
	\vspace*{\fill}
	{
     \large \MetaDate\ (Version \MetaVersion)
	}
\end{titlepage}


%---------------------------------
% EINLEITUNG
%---------------------------------
\section{Einleitung}
Diese Bachelorarbeit beschreibt eine mögliche Integration des User Centered Design (UCD)-Prozesses in agile Methode am Beispiel Scrum. 
Im Rahmen einer Überarbeitung der DIN EN ISO 9241 wurde der Prozess 2006 von „User Centered Design“ in „Human Centered Design“ umbenannt. 
In der Literatur wird jedoch noch häufig der Begriff „User Centered Design“ verwendet. Da beide Begriffe analog genutzt werden wurde auf die, zur Zeit noch, 
ungebräuchliche Verwendung des offiziellen Prozessnamens verzichtet. Im ersten Teil dieser Arbeit wird der UCD-Prozess und einige der darin verwendeten Methoden vorgestellt. 
Der zweite Teil erläutert den Scrum-Prozess. Im abschließenden dritten Teil wird versucht, eine mögliche Verbindung der beiden Prozesse 
zu zeigen und an Hand eines Beispiels erläutert. Für die Erstellung dieser Bachelorarbeit wurde der englische Scrum-Guide benutzt.
Die jeweiligen Begriffe für Meetings, Rollen, etc. wurden daher nicht übersetzt.
Die LH IND TS GmbH wurde durch einen Kunden beauftrag eine Analyseapplikation neu zu entwickeln. Der Kunde vertritt über 3000 aus metallverarbeitenden Unter-
nehmen. Auf Grund von Größe und Umfang des Projektes (geschätzt etwa 900 Personentage) wurde zusammen mit dem Auftraggeber vereinbart, das Projekt mit Scrum als Projektmanagement-Framework umzusetzen. 
Da das Produkt, neben dem technischen Teil (Verfügbarkeit und Performance),  für die Benutzer leicht zu bedienen sein sollte, wurde bereits in der Angebotsphase die
Integration des Bereiches UCD empfohlen. Der UCD-Bereich der Cocomore AG ist zuständig für die konzeptionelle Entwicklung von benutzerfreundlichen Bedienkonzepten. Der Kunde (Kunde) folgte dieser Empfehlung.


%---------------------------------
% PROBLEMSTELLUNG UND LÖSUNGSANSATZ
%---------------------------------
\section{\label{sec:problem_loesung}Problemstellung und Lösungsansatz}
 
Der Kunde hatte in der Vergangenheit mit mindestens drei verschiedenen Agentu-
ren ein Redesign seines E-Commerce-Angebots durchgeführt. Die Ergebnisse waren
nicht das was sich der Kunde erhofft hatte. Es sollte eine Performance-Steigerung
sowie eine Erhöhung der Besucherzahlen erreicht werden. Beide Anforderungen
konnten in den vorangegangenen Redesigns nicht erfüllt werden.
\\
In den vorangegangenen Redesigns wurde auf klassische Projektmanagement-
Frameworks zurückgegriffen. Der Kunde war von der Idee Scrum zu nutzen, und
den Entwicklungsprozess aktiv zu begleiten und mitzugestalten zu können, leicht zu
überzeugen. In den vorherigen Redesigns konnte der Kunde erst sehr spät Entwick-
lungsergebnisse überprüfen und aufgetretene Probleme zögerten eine Fertigstellung
immer hinaus.
\\
Ein weiterer Grund für den Einsatz von Scrum lag in dem großen Projektumfang
(geschätzt etwa 900 Personentage). Der zuständige Projektmanager, sowie der
Leiter des IT-Bereiches der Cocomore AG, schlugen dem Kunden auch aus diesem
Grund Scrum als Projektmanagementtool vor. Durch die Verwendung von Scrum
konnte der Kunde ständig den Fortschritt des Projektes überprüfen und bei Pro-
blemen eingreifen. Der Einsatz linearer Entwicklungsmethoden hätte aus Sicht des Projektmanagers und des IT-Leiters evtl. 
zu einer Wiederholung der Probleme aus den vorangegangenen Redesigns geführt.


%---------------------------------
% PROBLEMSTELLUNG
%---------------------------------
\subsection{Problemstellung des }
\subsubsection{Was ist Human Centered Design?}

Human Centered Design (UCD) ist ein Prozess um eine möglichst hohe positive User Experience (Nutzungserlebnis) zu erreichen. Die User Experience wird durch die Usability in starkem Masse beeinflusst.
\\
Bei der Konzeption von Produkten nach UCD wird der Benutzer in den Mittelpunkt
des Entwicklungsprozesses gestellt. Es wird dabei versucht, den Benutzer während
der gesamten Entwicklung eines Projektes in den Entwicklungsprozess einzubinden
und die Benutzeroberfläche und Interaktionen so zu optimieren, dass die fertige
Anwendung den Nutzer in seiner Aufgabe optimal unterstützt.
\\
Der äußere orange Rahmen symbolisiert die Anforderungen an einen benutzerori-
entierten Entwicklungsprozess (User Centered Design). Der grüne Rahmen steht
für das Usabiltiy Engineering, das einen Ausschnitt des User Centred Design bildet.
Usability Engineering konzentriert sich zwar auf das Überprüfen und Optimieren
der Usability des Produktes („während der Nutzung“), ist jedoch nicht unabhängig davon, was vor und nach der Nutzung passiert: bereits generierte Erwartungen und
Vorerfahrungen sowie während der Nutzung neu gewonnene Erfahrungen spielen ebenfalls eine wichtige Rolle. User Centered Design bietet den Rahmen in dem die Methoden des Usability En-
gineering angewandt werden.

\subsubsection{Die vier Phasen des User Centered Design}

\begin{enumerate}
	\item Analyse des Nutzungskonzepts
	\item Definition der Anforderungen
	\item Konzeption und Entwurf
	\item Evaluation
\end{enumerate}

\subsubsection{Agile Entwicklungsmethoden}

Agile Entwicklungsmethoden basieren auf der Idee, durch häufige Iterationen die
Entwicklungszeit für Projekte zu verkürzen und ein flexibles System zur Problem-
lösung zu benutzen. Anders als bei linearen Entwicklungsmethoden wird ein zu
entwickelndes System nicht vollständig geplant bevor die eigentliche Umsetzung
erfolgt. Viel mehr ist das Ziel agiler Entwicklung, dem Kunden nach jeder Iteration
ein funktionierendes Produkt zu liefern.
Der Kunde wird bereits sehr früh in den Entwicklungsprozess eingebunden und
begleitet diesen bis zum fertigen Produkt.
\\
Seit den Anfängen der agilen Entwicklung haben sich verschiedene Methoden eta-
bliert, einige der bekanntesten sind

\begin{itemize}
	\item Extreme Programming
	\item Feature Driven Development 
	\item Crystal (oder auch Methodensammlung)
	\item Usability Diven Debelopment 
	\item Scrum
\end{itemize}

\noindent
\\
Agile Entwicklungsmethoden bieten einige Vorteile, sowohl für die Entwickler, als
auch für den Kunden:

\begin{itemize}
	\item Agile Entwicklungsmethoden lassen sich auf nahezu alle Projekte anwenden
	\item Frühe Beteiligung des Kunden am Entwicklungsprozess
	\item Nach jeder Iteration erhält der Kunde ein funktionsfähiges Produkt
	\item Mehr Freiheit für den Entwickler bei der eigentlichen Umsetzung / Problemlösung
	\item Auf Grund der Flexibilität können Änderungen am Projekt schnell und meist auch kostengünstig realisiert werden.
\end{itemize}
\noindent
\\
Die bei linearen Entwicklungsmethoden teilweise sehr aufwendigen Iterationen wer-
den von Anfang an in den Prozess integriert und dienen der dynamischen Entwick-
lung eines Projektes.
\noindent



%%---------------------------------
%% LÖSUNGSANSATZ
%%---------------------------------
\subsection{Lösungsansatz}

In diesem Abschnitt wird zusammengefasst, was bei der Lösung des zu bearbeitenden \qto{Problems} -- nämlich der Erstellung eines Exposés \bzw der Anfertigung einer Bachelor- oder Masterarbeit -- in inhaltlicher, methodischer und formaler Hinsicht zu beachten ist. Weitere Informationen zum wissenschaftlichen Arbeiten finden sich in entsprechenden Ratgebern (siehe \zB~\cite{lit:FolzEtAl:Studi-SOS}).

\paragraph{Inhaltliches: Was ist eigentlich eine geeignete Problemstellung?}

Problemstellungen für eine Bachelor- oder Masterthesis können durch Ausschreibungen eines Professors gegeben sein, aus persönlichem Interesse erwachsen oder -- wie häufig anzutreffen -- aus einem Unternehmen stammen, bei dem man gerade sein Praktikum abgeschlossen hat. Gerade bei letzteren ist es häufig der Fall, dass die vorliegenden Probleme nur auf die Belange des Unternehmens zugeschnitten und damit sehr speziell sind (\qto{Implementieren Sie doch noch Modul X für unsere Anwendung!}). Zudem bieten entsprechende Probleme in konzeptioneller Hinsicht oftmals wenig Spielraum.

Eine wichtige Anforderung an ein geeignetes Thema für eine Thesis ist, dass ein hinreichend \emph{allgemeines} Problem betrachtet wird, dessen Lösung dann für viele interessant sein kann. Ein solches allgemeines Problem kann vielfach ausgehend von einem speziellen Fall (\zB aus dem erwähnten Praktikum) durch Abstraktion, \dH Generalisierung gewonnen werden. Im Rahmen der Arbeit sollte dann dieses generalisierte Problem analysiert und ein möglichst allgemeines Konzept entwickelt werden, das dann, \zB unter Verwendung einer bestimmten Plattform und \ggf wieder eingegrenzt auf das speziellere Problem, im Sinne eines \qto{Proof of Concepts} prototypisch umgesetzt werden kann, um letztendlich zu zeigen, dass das (allgemeine) Konzept funktioniert.

Sie (und \evtl beteiligte Unternehmen) sollten aber davon ausgehen, dass das Ergebnis einer Bachelor- oder Masterarbeit typischerweise kein fertiges Produkt ist, das im Unternehmen direkt eingesetzt werden kann. In der Regel ist es eher so, dass auf Basis der Thesis und der prototypischen Implementierung anschließend ein System entwickelt werden kann, das auch produktiv eingesetzt werden kann.

\paragraph{Methodisches: Wie gehe ich denn nun vor?} Die Bachelor- oder Masterthesis zeigt ihre Fähigkeit, eine Aufgabe mit wissenschaftlichen Mitteln in begrenzter Zeit zu bearbeiten. Ein wissenschaftliches Vorgehen sieht immer einen \textit{Analyseteil} (\qto{Was sind die Anforderungen, welche Rahmenbedingungen sind zu beachten, was gibt es für Möglichkeiten?}) und einen \textit{Syntheseteil} (\qto{Was habe ich weshalb wie gemacht?}) vor. Im Folgenden ein grober Gliederungsrahmen als Beispiel\footnote{Ein etwas konkreteres Muster für eine (generische) Gliederung findet sich in \cref{sec:gliederung}.}:

\begin{gliederung}
	\item Einleitung: Hier führt eine kurze Beschreibung des Problemfeldes \bzw der Motivation, der konkreten Aufgabe(n), der angestrebten Vorgehensweise und \ggf auch ein erster Blick auf mögliche Lösungsansätze in das Thema der Arbeit ein (\qto{Ziel der Arbeit ist ...}). Danach sollte noch ein Abschnitt folgen, der kurz in Prosa die weitere Struktur der Arbeit und die wesentlichen Inhalte der Kapitel darstellt. Danksagungen (und Verwünschungen) gehören ins Vorwort.
	
	\item Analyse: Der \qto{Fleißteil} mit viel Literaturrecherche, \zB zu Fragestellungen wie
	\begin{itemize}
		\item Problemanalyse: Welche Anforderungen sollen erfüllt werden?
		\item Verwandte Arbeiten: Welche alternativen Lösungsansätze gibt es schon in diesem Problemfeld? Welche Stärken und Schwächen haben diese Ansätze?
		\item Grundlagen: Welche Konzepte und Methode(n) sind für diese Arbeit relevant? Welche (Software-)Werkzeuge können benutzt werden?
	\end{itemize}
	Der Analyseteil ist \ggf in mehrere Kapitel unterteilt.

	\item Synthese: Hier geht es um das eigene \qto{Werk}: Konzeption und Realisierung, gerne in zwei Kapiteln (in dieser Reihenfolge!).
	\begin{itemize}
		\item Konzeption: Abstrakte, aber dennoch möglichst klare Spezifikation Ihres Lösungsansatzes. Mithilfe von Text und Diagrammen (\zB zu Interaktionsstruktur, Software-Architektur, Datenbankschema) wird ein Modell Ihrer Lösung (Struktur und Dynamik) beschrieben und darüber hinaus die zu deren Erstellung verwendeten Methoden.
		\item Realisierung (natürlich nur, wenn was realisiert wurde): Beschreibung ausgewählter und interessanter Aspekte einer konkreten Umsetzung Ihres Konzepts auf einer (Programmier-)Plattform. Dazu können beispielsweise der Einsatz von besonderen Werkzeugen, Optimierungen oder spezielle Problemlösungen gehören, die aufgrund der gewählten Plattform erforderlich werden und daher nicht bereits durch das Konzept abgedeckt sind. 

	\end{itemize}
		
	\item[] Ein gutes Konzept zeichnet sich vielfach dadurch aus, dass der Realisierungsteil eher klein bleiben kann, weil die Realisierung ja eben \qto{nur} die Umsetzung eines Konzepts in einer Programmierplattform darstellt und daher nicht jedes Detail aufgeführt werden muss (dafür gibt es als Anlage eine CD/DVD). 
%	
	Dennoch gilt: Wissenschaftliche Ergebnisse müssen reproduzierbar sein. Daher ist es wichtig, dass die Arbeit möglichst alle relevanten Details für eine Re"=Implementierung und Reproduktion der Ergebnisse enthält.
	
	\item Evaluation: 
	Teil einer wissenschaftlichen Arbeit ist die kritische Überprüfung und Diskussion der erzielten Ergebnisse (Konzept und Realisierung), wenn möglich auch der Vergleich mit bestehenden Ansätzen.
	Welche Aspekte dabei im Vordergrund stehen, hängt vom Themenfeld Ihrer Arbeit ab: Performance einer Anwendung, Usability eines UI, Sicherheit einer Schnittstelle, Flexibilität einer Software-Architektur... Machen Sie zum Abschluss Ihrer Arbeit deutlich, welche Ziele Sie zu welchem Grad erreicht haben (auch wenn vielleicht nicht alles zum Besten steht).

	\item Zusammenfassung und Ausblick: Das Schlusskapitel führt die wichtigsten Ergebnisse der Arbeit auf und zeigt, wo es weitergehen kann oder was noch verbessert werden müsste. Ein (eiliger) Leser von ausschließlich Einleitung sowie Zusammenfassung und Ausblick sollte bereits eine Vorstellung davon haben, welche Ziele die Arbeit verfolgt hat und welche davon inwieweit erreicht wurden. 

\end{gliederung}

\noindent
Wie bereits in den Anmerkungen zum Analyseteil angedeutet, kommt der Literaturrecherche eine besondere Bedeutung zu: Welche verwandten Arbeiten gibt es? Wo finde ich Informationen zu relevanten Konzepten, Methoden oder Technologien? Entsprechende Literatur ist zu recherchieren und -- im kleineren Umfang -- bereits im Exposé zu referenzieren.

Recherchearbeit ist ein oftmals aufwändiger und langwieriger Prozess, bei dem man in der Regel erst allmählich durch das Lesen vieler Quellen und das Verfolgen von Referenzen zu \qto{guter} Literatur gelangt. Lassen Sie sich nicht verleiten, sich auf bequem erreichbare Online-Quellen wie Blogs oder Wikipedia zu verlassen. Auch wenn Wikipedia immerhin eine Art Qualitätssicherung durch die \qto{Crowd} bietet und teilweise auch für den Einstieg brauchbare Ergebnisse liefert, stellen derartige Quellen in aller Regel Informationen zur Verfügung, die nicht objektiv von Fachexperten auf ihre Richtigkeit geprüft wurden (Sie werden im Web auch viele Quellen finden, die scheinbar glaubhaft \qto{belegen}, dass es keinen Klimawandel gibt oder die Evolution nie stattgefunden hat).
%
Online-Quellen -- sofern es sich dabei nicht um (auch) im Netz auffindbare wissenschaftliche Dokumente wie Konferenzbeiträge oder technische Berichte mit Autoren, Titel, Organisation und Datum handelt -- sind keine Literatur im engeren Sinne und sollten (sofern sie denn verwendet werden) bevorzugt in einem gesonderten Verzeichnis gelistet werden. 
%
Besser ist es in jedem Fall, sich auf Fachbücher \cite{lit:GammaEtAl:DesignPatterns} und wissenschaftliche Beiträge \cite{lit:SyromiatnikovEtAl:Model-View-*DesignPatterns} zu Konferenzen zu konzentrieren. Hier kann man davon ausgehen, dass diese Publikationen einen Review-Prozess durchlaufen haben und somit durch Fachexperten geprüft und für (hinreichend) gut befunden wurden.

Für die Recherche gibt es eine Reihe von Quellen, die Sie als Studierende der Hochschule Bremen nutzen können:
\begin{itemize}
	\item Staats- und Universitätsbibliothek Bremen (SuUB)\footnote{\url{https://www.suub.uni-bremen.de}}, 
	\item digitale Bibliotheken von Fachgesellschaften wie die ACM Digital Library\footnote{\url{https://dl.acm.org}} und IEEE XPlore Digital Library\footnote{\url{https://ieeexplore.ieee.org/Xplore/home.jsp}}			\item Google Scholar\footnote{\url{https://scholar.google.de}}
\end{itemize}

\noindent
Sofern Sie im Hochschulnetz sind oder sich von zuhause über das VPN der Hochschule Bremen mit dem Hochschulnetz verbunden haben, können Sie online verfügbare Texte der SuUB runterladen und auch die genannten digitalen Bibliotheken kostenlos nutzen.

%\pagebreak
Für die Literaturverwaltung bieten sich Werkzeuge wie \zB \emph{Mendeley}\footnote{\label{footnote:mendeley}\url{https://www.mendeley.com/}} oder \emph{Zotero}\footnote{\url{https://www.zotero.org}} an. Diese bieten neben der Möglichkeit zur Verwaltung von Literatur in der Cloud und den Export von bibliographischen Einträgen in verschiedene Zielformate weitere Features wie \zB die Möglichkeit zur Suche nach Literatur innerhalb des Werkzeugs.

Von besonderer Bedeutung in einer wissenschaftlichen Arbeit ist es, dass Sie Ihre Bewertungen (\zB von Lösungsansätzen) grundsätzlich deutlich machen und daraus resultierende Entscheidungen für den Leser nachvollziehbar begründen, sowohl in der Analyse als auch in der anschließenden Konzeption und Realisierung.
%: Ergebnisse fallen nicht vom Himmel, auf Ihre Argumentation kommt es an! 
Stellen Sie sich dazu stets Fragen Ihres Gutachters wie \qto{Warum haben Sie das so gemacht?} oder \qto{Welche Alternativen haben Sie betrachtet?} und beantworten Sie die Fragen in der Arbeit (und nicht erst im Kolloquium).


\paragraph*{Formales: Was muss ich alles beim Schreiben eines wissenschaftlichen Textes beachten?} An wissenschaftliche Texte werden eine Reihe von formalen Anforderungen gestellt, die letztendlich das Ziel haben, die Lesbarkeit und Verständlichkeit des Textes zu erhöhen \cite{lit:Prevezanos:TechnischesSchreiben}. Einige davon werden hier aufgeführt.

\noindent
Schreibstil:
	\begin{itemize}
		\item Wissenschaftliche Distanz: Schreiben Sie nicht in der ersten Person (also nicht \qto{ICH habe mich dafür entschieden, dies und das so und so zu machen.}), sondern wählen Sie stattdessen lieber eine passive, distanzierte Form (\qto{Im Rahmen dieser Arbeit wurde entschieden, ...} oder \qto{Die Entscheidung, ..., war begründet durch ...}).
		\item Inhaltliche Übernahmen aus der Literatur müssen selbstverständlich durch entsprechende Quellenangaben kenntlich gemacht werden. 
		\item Verzichten soweit wie möglich auf Zitate, sondern finden Sie Ihre eigenen Worte für Inhalte, die Sie aus der Literatur übernehmen. Zitate bieten sich nur dann an, wenn die wörtliche Wiedergabe von Inhalten von Bedeutung ist (und das ist zumeist nicht der Fall). Zitate müssen durch Anführungszeichen und Quellenangabe kenntlich gemacht werden.
		\item Plagiate sind jedwede Übernahme von Inhalten anderer, die \emph{nicht} als solche gekennzeichnet sind. Verstoßen Sie dagegen, führt das zwingend zum Nichtbestehen, \ggf sogar zur Exmatrikulation\footnote{Siehe jeweils gültige Fassung der Bachelor- und Masterprüfungsordnung (Allgemeiner Teil) unter\\ \url{http://www.hs-bremen.de/internet/de/hsb/hip/dokumente/po/}.}. Also: nutzen Sie das (gesicherte) Wissen anderer, aber weisen Sie unbedingt auf die Quellen hin.
		\item Prüfen Sie Ihre Rechtschreibung! Und falls Sie da nicht über das richtige Händchen verfügen sollten: es gibt eine Reihe von Programmen mit einer automatischen Rechtschreibüberprüfung. Keine Ausreden!
		\item Viele Seiten sind kein Zeichen von Qualität! Gewürdigt wird, wenn das Wesentliche kompakt dargestellt und auf den Punkt gebracht wird, ohne dabei die Nachvollziehbarkeit zu beeinträchtigen. Das kann \zB durch die Darstellung von Zusammenhängen durch selbst erstellte Grafiken, Übersichtstabellen und Spiegelstriche statt Prosa \emph{unterstützt} werden. 
		
		Als geeigneter Umfang für eine Bachelorarbeit haben sich 40-60 Seiten gezeigt, bei einer Masterarbeit 60-80 Seiten (jeweils ohne Anhänge).
	\end{itemize}
	
\noindent
Referenzen:	
	\begin{itemize}
		\item \emph{Jede} Abbildung, Tabelle und jedes Listing etc.\ bekommt eine Kategorie, Nummer und Über- oder Unterschrift (\zB \emph{Listing 1: Hocheffizienter Suchalgorithmus}). 
		Da Tabellen und Listings über eine Seite hinausgehen können, werden ihre Beschriftungen oberhalb der Tabelle \bzw des Listings angegeben, um frühzeitig die Bedeutung der Tabelle \bzw des Listings deutlich zu machen. Abbildungen werden unterhalb beschriftet.
		\item \emph{Jede} Abbildung, Tabelle und jedes Listing etc.\ muss mindestens einmal im Text referenziert werden und sollte in diesem Zusammenhang zudem im Text diskutiert werden, um die wesentliche Botschaft deutlich zu machen. (\zB \qto{\cref{lst:bubblesort} zeigt \emph{<Botschaft>}} oder \qto{\emph{<Diskussion...>} (siehe \cref{lst:bubblesort})}). 
		\item \emph{Jede} Literaturquelle muss mindestens einmal im Text referenziert werden. Die Angabe der Seitenzahl ist in der Informatik unüblich, schadet aber auch nicht.
		\item Wenn Sie URLs zu Organisationen wie der Object Management Group (OMG)\footnote{\url{https://www.omg.org}}, Firmen wie \zB Apple\footnote{\url{https://www.apple.com/de/}} oder Produkten wie Mendeley\footref{footnote:mendeley} angeben wollen, geschieht das sinnvollerweise in einer Fußnote. Online-Quellen \cite{lit:Gossman:MVVM}, die \qto{lesbare} und mit Autor und Titel versehene Texte bezeichnen, gehören dagegen ins Literaturverzeichnis.
	\end{itemize}
	
%%% Listing
\lstinputlisting[firstnumber=1,caption={Hocheffizienter Suchalgorithmus}, label={lst:bubblesort}]{bubblesort.tex}
%%% Listing


\noindent
Struktur:
	\begin{itemize}
	 	\item Ein neues \emph{Kapitel} (nicht Abschnitt oder Unterabschnitt) in einer Bachelor- oder Masterarbeit beginnt üblicherweise auf einer neuen Seite. Bei beidseitigem Layout beginnt ein neues Kapitel zudem auf einer ungeraden (rechten) Seite.
	 	\item Nach einer Überschrift folgt nicht direkt eine zweite (Unter-)Überschrift, sondern zumindest eine kurze Beschreibung dessen, was in den folgenden Unterabschnitten zu erwarten ist (vgl.~\cref{sec:problem_loesung}).
	 	\item Ein Kapitel oder Abschnitt mit weiterer Untergliederung enthält mindestens zwei Unterabschnitte. Oder anders formuliert: wenn es einen Abschnitt x.y.1 gibt, dann gibt es auch einen Abschnitt x.y.2 (ansonsten gibt es eben nur Abschnitt x.y)!
	 	\item Verzichten Sie auf Vorwärtsreferenzen! Es ist im Allgemeinen schlechter Stil, bereits in Kapitel 2 mittels eines \qto{siehe Abschnitt 4.3.1} auf Ergebnisse zu verweisen, die erst später folgen (Ausnahmen bestätigen die Regel).
	\end{itemize}


\noindent
Sonstiges:
\begin{itemize}
	\item Sehr spezifische Fachbegriffe sollten in einem Glossar mit ein bis zwei Sätzen erklärt werden. Hilft auch dem Schreiber! Neben Literaturverzeichnis und Glossar sind noch Verzeichnisse für Abbildungen, Tabellen und Listings gebräuchlich.
	\item Wenn Abkürzungen verwendet werden sollen, muss die Abkürzung gemeinsam mit dem abzukürzenden Begriff explizit eingeführt werden. Also erst \qto{Asynchronous JavaScript and XML (AJAX) ist ein ...}, bevor später dann nur noch mit \qto{AJAX} gearbeitet wird. Ein Abkürzungsverzeichnis kann unterstützen, ersetzt aber die explizite Einführung der Abkürzung im Text nicht.
 	\item Code-Fragmente bitte nicht mittels Screenshot als Abbildung einbinden. Das ist zwar einfach, jedoch ist das Resultat oft optisch unschön und -- und das ist wichtiger -- spätestens im Ausdruck in der Regel nicht mehr lesbar (und auch nicht durchsuchbar). Fügen Sie Code-Fragmente wie in \cref{lst:bubblesort} gezeigt als 
 	%als Listing 
 	formatierte Texte ein.
	\item Skalieren Sie Abbildungen gleichen Typs (\zB UML-Diagrammen) oder mit gleichen Fonts einheitlich. Ggf.{\xspace} erfordert das dann etwas Arbeit am Layout, um eine Abbildung mit vorgegebener Skalierung auf die Abmessungen einer Seite anzupassen.
 	\item Zahlen bis einschließlich zwölf schreibt man üblicherweise aus. Also \qto{die sieben Zwerge}, aber \qto{die wilde 13}!
	\item \emph{LaTeX} \cite{lit:Schlosser:WissArbeitenMitLatex} ist schön und gerade im Wissenschaftsbetrieb der Informatik verbreitet, aber nicht Pflicht. 
	
	Die Erstellung des Exposés stellt grundsätzlich eine gute Möglichkeit dar, sich in das Textverarbeitungssystem der Wahl einzuarbeiten und dabei auch Features wie automatische Nummerierung (\zB von Kapiteln, Abbildungen), Querverweise, Erstellung von Verzeichnissen, Literaturverwaltung etc.\ als Vorbereitung auf die Thesis praktisch anzuwenden.
\end{itemize}



\noindent
Und zu guter Letzt noch ein wichtiger Gedanke: Immer an den Leser denken! Der weiß noch nicht alles, was Sie wissen (oder im Laufe der Arbeit gelernt haben). Versuchen Sie daher, sich in seine Rolle hineinzuversetzen und führen Sie ihn möglichst so durch den Text, dass er Ihre Gedankengänge nach und nach verstehen kann. 


\pagebreak

%---------------------------------
% KONKRETE AUFGABEN
%---------------------------------
\section{Konkrete Aufgaben}

Für die Erstellung eines Exposés sind die folgenden Aufgaben durchzuführen:

\begin{itemize}
	\item Recherche: Ermittlung von Literatur zum wissenschaftlichen Arbeiten, Vergleich mit ähnlichen Dokumenten
 	\item Anforderungsanalyse: Zusammenstellung von Anforderungen an das zu erstellende Exposé \bzw die Bachelor- oder Masterarbeit.
 	\item Konzeption: Entwicklung eines Konzepts für das Exposé / die Thesis: Textverarbeitung / Textsatzsystem, Gliederung, Konventionen, ...
 	\item prototypische Implementierung: Schreiben des Exposés \bzw der Thesis
 	\item Evaluation: Bewertung und Diskussion der Ergebnisse (machen auch die Prüfer ;-) )
 	\item Qualitätssicherung: sorgfältige Überprüfung des Dokuments auf inhaltliche und formale Fehler
\end{itemize}


%---------------------------------
% VORLÄUFIGE GLIEDERUNG
%---------------------------------
\section{\label{sec:gliederung}Vorläufige Gliederung}

{\small
\emph{Hinweis: Im Folgenden wird exemplarisch eine \emph{generische} Gliederung gezeigt, die selbstverständlich abhängig vom Thema angepasst werden muss.}
}

{\parindent=5mm 
Eigenständigkeitserklärung

Zusammenfassung / Abstract}

\begin{gliederung}
	\item Einleitung
	\begin{gliederung}
		\item Problemfeld
		\item Ziele der Arbeit
		\item Lösungsansatz
		\item Aufbau der Arbeit
	\end{gliederung}

	\item Anforderungsanalyse
	\begin{gliederung}
		\item Funktionale Anforderungen
		\item Nicht-funktionale Anforderungen
		\item Zusammenfassung
	\end{gliederung}
	
	\item Grundlagen und verwandte Arbeiten
	\begin{gliederung}
		\item User Centered Design
			\subitem 
		\item Grundlagenthema 2
		\item ...
		\item Verwandte Arbeiten
	\end{gliederung}
	
	\item Konzeption
	\begin{gliederung}
		\item Entwurf einer Software-Architektur
		\item Design des User Interfaces
		\item Entwurf der Komponenten
		\begin{gliederung}
			\item Komponente 1
			\item Komponente 2
			\item ...
		\end{gliederung}
		\item Spezifikation der Schnittstellen
		\item Zusammenfassung
	\end{gliederung}
	
	\item Prototypische Realisierung
	\begin{gliederung}
		\item Wahl der Realisierungsplattform
		\item Festlegung des Realisierungsumfangs
		\item Ausgewählte Realisierungsaspekte
		\begin{gliederung}
			\item REST-API mit Bibliothek XYZ
			\item ...
		\end{gliederung}
		\item Qualitätssicherung
		\item Zusammenfassung
	\end{gliederung}

	\item Evaluation
	\begin{gliederung}
		\item Überprüfung funktionaler Anforderungen
		\item Überprüfung nicht-funktionaler Anforderungen
		\item Zusammenfassung
	\end{gliederung}

	\item Zusammenfassung und Ausblick
	\begin{gliederung}
		\item Zusammenfassung
		\item Ausblick
	\end{gliederung}

\end{gliederung}

{\parindent=5mm
Anhang}
\begin{itemize}%[itemindent=6mm]
\item[A] Anhang <Thema A>
\item[B] Anhang <Thema B>
\item[...]
\end{itemize}

{\parindent=5mm
Literaturverzeichnis}%\\[0.25cm]
% Vertikaler Abstand
%\vspace*{\baselineskip}
%\smallskip  % kleiner variabler Abstand
\medskip    % mittlerer variabler Abstand
%\bigskip    % großer variabler Abstand

\pagebreak
\noindent
Ergänzende Anmerkungen:\nopagebreak
\begin{itemize}
	\item Wenn die Evaluation eher knapp ausfällt, kann sie \ggf auch zu einem Abschnitt des Kapitels 5 (Prototypische Realisierung) werden.
	\item Die eigentliche (nummerierte) Gliederung wird noch durch Eigenständigkeitserklärung, Zusammenfassung / Abstract, Verzeichnisse und \evtl Anhänge ergänzt:
	\begin{itemize}
		\item In der Eigenständigkeitserklärung versichern Sie unter anderem, die Arbeit selbständig, ohne unzulässige fremde Hilfen und unter Angabe aller verwendeten Quellen und Hilfsmittel angefertigt zu haben. 
		Das Prüfungsamt der Hochschule Bremen stellt auf seiner Web-Seite ein Muster\footnote{\urlMitUmlauten{https://www.hs-bremen.de/mam/hsb/dezernate/d3/erkl\%C3\%A4rung\_\%C3\%BCber\_das\_eigenst\%C3\%A4ndige\_erstellen\_der\_arbeit.doc}{https://www.hs-bremen.de/mam/hsb/dezernate/d3/erklärung\_über\_das\_eigenständige\_erstellen\_ der\_arbeit.doc}} bereit, das allerdings noch angepasst werden muss.
		\item Zusammenfassung \bzw englischer Abstract enthalten jeweils nur einen Abschnitt, der den Inhalt der Arbeit und die erzielten Ergebnisse ohne gesonderte Einführung in das Thema zusammenfasst.
		\item Neben dem Literaturverzeichnis sind noch Verzeichnisse für Abbildungen, Tabellen, Listings und Abkürzungen gebräuchlich. Ein Glossar kann helfen, spezielle und dem Leser im Allgemeinen nicht bekannte Fachbegriffe zu erläutern.
		\item Anhänge können dazu dienen, der Arbeit wichtige Inhalte beizufügen, die für die Wiedergabe im Hauptteil nicht geeignet erscheinen, weil Sie \zB nicht für jeden Leser relevant sind oder allgemein den Lesefluss stören.
	\end{itemize}
\end{itemize}


%---------------------------------
% ZEITRAHMEN
%---------------------------------
\section{Zeitplanung}

Geplanter Starttermin\footnote{Hinweis: Das Prüfungsamt erbittet sich 14 Tage Vorlauf. Dementsprechend muss der Starttermin ausgehend vom Zeitpunkt der Anmeldung einer Bachelor- oder Masterarbeit mindestens zwei Wochen in der Zukunft liegen.}: 
1. September 2018

\noindent
Bearbeitungsdauer: 9 Wochen (bei Bachelorarbeiten) / 22 Wochen (bei Masterarbeiten)

\noindent
\cref{tab:zeitplanung} stellt die geplanten Arbeitspakete und Meilensteine dar:
%
\begin{table}[H]		% 'H' option is provided by float package 

\caption{Arbeitspakete und Meilensteine} \label{tab:zeitplanung} 

\centering
\def\arraystretch{1.3}%  1 is the default, change whatever you need
%\setlength{\extrarowheight}{5pt}%
\begin{tabular}{|c|p{10cm}|c|}

\hline 
%%%%%%
 M1 & Offizieller Beginn der Arbeit & 01.09.2018  \\ 
\hline
 & \tabitem Einrichtung der Textverarbeitung & \multirowcell{3}{1,5 Wochen} \\ 
 & \tabitem Anforderungsanalyse & \\ 
%\cline{2-2}
 & \tabitem Verfassen der Thesis: Kapitel 2 (Anforderungsanalyse) & \\ 
\hline
%%%%%%
 M2 & Abschluss der Analysephase & 11.09.2018  \\ 
\hline
 & \tabitem Recherche & \multirowcell{2}{1,5 Wochen} \\ 
%\cline{2-2}
 & \tabitem Verfassen der Thesis: Kapitel 3 (Grundlagen) & \\ 
\hline
%%%%%%
 M3 & Abschluss der Recherchephase & 21.09.2018  \\ 
\hline
 & \tabitem ... & \multirowcell{2}{n Wochen} \\ 
%\cline{2-2}
 & \tabitem Verfassen der Thesis: ... & \\ 
\hline
%%%%%%
 M4 & Abschluss der ...phase & ...  \\ 
\hline
\multicolumn{3}{|c|}{...} \\
\hline
%%%%%%
 M$y$ & Erste Fassung vollständige Thesis & 27.10.2018  \\ 
\hline
 & \tabitem Korrekturlesen & \multirowcell{2}{1 Woche} \\ 
%\cline{2-2}
 & \tabitem Drucken / binden lassen & \\ 
\hline
%%%%%%
 M$z$ & Abgabe der Thesis & 02.11.2018  \\ 
\hline


\end{tabular} 
\end{table}


%---------------------------------
% PERSONEN
%---------------------------------
\section{Unterschriften}

Student\_in:

 Name: \textit{<Vor- und Nachname>}
 
 Adresse: \textit{<Straße und Hausnummer, PLZ, Stadt>}
 
 Telefon: \textit{<Nummer, unter der man Sie bei wichtigen Fragen schnell erreichen kann.>}
  
 E -Mail: \textit{<mail@stud.hs-bremen.de>}


 \hfill \rule{7cm}{0.2mm}

 \hfill Unterschrift (Student\_in)
 
\noindent
Erstgutachter\_in / Betreuer\_in: \textit{<Prof.~Dr. Ernst>} \\[0.2cm]

 \hfill \rule{7cm}{0.2mm}

 \hfill Unterschrift (Erstgutachter)


\noindent
Zweitgutachter\_in: \textit{<Prof.~Dr. Lustig>}




%---------------------------------
% LITERATUR
%---------------------------------
%\bibliographystyle{IEEEtran}
%\bibliography{IEEEabrv,literatur}
%\bibliographystyle{plain}
%\bibliography{literatur}


% Gesamte Literaturliste
%\printbibliography	

% Literaturliste getrennt in Papier- und Online-Quellen:
\printbibheading
\printbibliography[nottype=online, heading=subbibliography, title={Gedruckte Quellen}]
\printbibliography[type=online, heading=subbibliography, title={Online-Quellen}]


\end{document}
